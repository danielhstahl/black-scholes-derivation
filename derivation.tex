\documentclass{article}
\usepackage{amsmath}
\usepackage{fancyhdr}
\usepackage{amsthm}
\usepackage{amsfonts}
\usepackage{cite}
\usepackage{float}
\usepackage{scrextend}
\usepackage{hyperref}
\usepackage{caption}
\usepackage{subcaption}

\setlength{\parindent}{0pt}
\newtheorem{theorem}{Theorem}
\begin{document}
\section{Simple derivation of Black Scholes}

The goal of this paper is to provide a rigorous, simple derivation of the Black Scholes formula.  The only pre-requisite mathematical and economic theory is basic (continuous) stochastic differential equations and the economic notion of efficiency.  Enumerating these pre-requisites:
\begin{enumerate}
	\item I use stochastic calculus and regard the existence of solutions to integrals such as \(dS_t=\alpha(S, t)dt+\sigma(S, t)dW_t\) to be established and rigorous (as it is).
	\item I use the economic concept that if two payoffs are equivalent, then economic agents will be indifferent between them and the existence of a market will force the two payoffs to have the same time \(0\) price.
\end{enumerate}

Note I do not use Feynman-Kac, Girsonov, or the First Fundamental Theorem of Asset Pricing.  However, I do need Girsonov for the derivation, so in the next section I prove it.

\section{Girsonov Proof}

\begin{theorem}
	Let \( Z_t=e^{-\int_0 ^t \theta_s dW_s-\frac{1}{2} \int_0 ^ t \theta^2 _s ds} \) where \(dW_s\) is an increment of (standard) Brownian Motion and \(\theta\) is measurable by the filtration generated by \(W_t\) and satisfies the ``usual'' conditions.  Then \(d\mathbb{\tilde{P}}_t=Z_t d\mathbb{P}_t\) is a probability measure. Further, \(d\tilde{W}_t=dW_t+\theta_t dt\) is a standard Brownian Motion under \(\mathbb{\tilde{P}}\).
\end{theorem}
\begin{proof}
The proof requires two parts.  Proof that \( d \mathbb{ \tilde{P} } \) is a probability measure:
\\
\\
A probability measure requires that \( \int_\Omega d \mathbb{ \tilde{P} }=1\).  Since \( \mathbb{E}[Z_T]=1\), this is self-evident.  Since \(d\mathbb{P}\) is a probability measure (by assumption), then \(d\mathbb{\tilde{P}}\) is a probability measure if it is an equivalent measure to \(d\mathbb{P}\).  Note that equivalence is a stronger statement, but by proving it I prove the weaker statement that \(d \mathbb{\tilde{P}}\) is a probability measure.  Equivalence is also immediately evident when noting that \(Z_t\) is almost-surely positive.  
\\
\\
It remains to prove that \(d\tilde{W}_t=dW_t+\theta_t dt\) is a standard Brownian Motion under \(\mathbb{\tilde{P}}\).

\[\mathbb{\tilde{E}}\left[e^{ui\tilde{W_T}}\right]=\mathbb{E}\left[Z_Te^{ui \left(\int_0 ^ T dW_t+\int_0^T \theta_t dt\right)}\right]\]
\[=\mathbb{E}\left[e^{\int_0 ^T (ui-\theta_t)dW_t-\frac{1}{2}\int_0^T \theta^2_t dt + \int_0^T ui\theta_t dt}\right]\]
Completing the square, 
\[=\mathbb{E}\left[e^{\int_0 ^T (ui-\theta_t)dW_t-\frac{1}{2}\int_0^T  (ui-\theta_t)^2 dt + \frac{1}{2} (ui)^2 T } \right]\]
\[=\mathbb{E}\left[ e^{ -\frac{1}{2} u^2 T } \right]\]

Recognizing this as the characteristic function of a standard Brownian Motion, the proof is complete.
\end{proof}

This proof implies that for any measurable function \(g\), \(\mathbb{E}[Z_T g(W_T)]\) can be written as \(\int_\Omega g(\omega) d\mathbb{\tilde{P}}(\omega)=\mathbb{\tilde{E}}[g(W_T)]\). 

\section{Derivation of Black Scholes}

Let there be a market in which two assets exist.  Canonically these two assets are a stock and a bond.  Let these markets be infinitely liquid so that there are buyers and sellers at any price and transaction costs are zero.  The goal is to construct a portfolio from these two assets that replicates the payoff of an option.  Mathematically, an option is a function \(h\) of one or more assets in the market.  For a European call option, \(h(x)=\mathbb{I}_{x>K}(x-K)\) where \(\mathbb{I}\) is the indicator function and \(K\) is the ``strike'' price.  
\\
\\
I now assume that the assets have the following dynamics:
\begin{equation}\label{eq1}
\begin{aligned}
dS_t = & \alpha(S_t, t)dt+\sigma(S_t, t)dW_t\\
dM_t = & r_t M_t dt
\end{aligned}
\end{equation}	

Besides the ``usual'' constraints on \(\alpha\) and \(\sigma\), I also require that these functions are chosen such that \(S_t\) is positive almost surely.  Additionally, \(r_t\) may be stochastic though it is measurable with respect to the Brownian Motion \(W_t\).

\begin{theorem}
	Let \ref{eq1} hold.  Then a contingent claim on \(S_t\) has the following price:
	\[v(S_t, t)=M_t\mathbb{E^M}\left[\frac{h(S_T)}{M_T} | \mathcal{F}_t\right]=S_t\mathbb{E^S}\left[\frac{M_T h(S_T)}{S_T}| \mathcal{F}_t \right]\]
	
\end{theorem}


\end{document}